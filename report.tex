\documentclass{article}

\usepackage{arxiv}

\usepackage[utf8]{inputenc}
\usepackage[T1]{fontenc}
\usepackage{hyperref}
\usepackage{url}
\usepackage{booktabs}
\usepackage{amsfonts}
\usepackage{amsmath}
\usepackage{amssymb}
\usepackage{nicefrac}
\usepackage{microtype}
\usepackage{cleveref}
\usepackage{graphicx}
\usepackage{natbib}
\usepackage{doi}
\usepackage{siunitx}

\title{\textbf{Range of Alpha Particles in Gas}}

\date{\today}

\author{Aidan Andrews\\
%   Department of Physics\\
%   University of Illinois at Urbana-Champaign\\
  \texttt{aidansa2}\\
  \And
  Tyler Wang\\
%   Department of Physics\\
%   University of Illinois at Urbana-Champaign\\
  \texttt{tylerww3}\\
}

\renewcommand{\headeright}{}
\renewcommand{\undertitle}{University of Illinois at Urbana-Champaign \\ PHYS 403}
\renewcommand{\shorttitle}{Range of Alpha Particles in Gas}

\hypersetup{
	pdftitle={Range of Alpha Particles in Gas},
	pdfsubject={Nuclear Physics, Alpha Particles, Energy Loss},
	pdfauthor={Aidan Andrews, Tyler Wang},
	pdfkeywords={Alpha Particles, Range, Energy Loss, Bethe Formula, Am-241},
}

\begin{document}
\maketitle

\begin{abstract}
    % TODO: Write after results/discussion are complete.
    % Should summarize: measured range of Am-241 alpha particles in air, He, Ar, N2
    % using solid-state detector + MCA. Compare to Bethe formula predictions.
    % Mention agreement/discrepancy with literature values.
\end{abstract}

% \keywords{Alpha Particles}

%===============================================================================
\section{Introduction}
\label{sec:intro}
%===============================================================================

% PURPOSE: Motivate the experiment, state what we're measuring, and give theoretical background.

% Alpha particles lose energy primarily through ionization and excitation of atoms in the medium.
% The moving charged particle exerts electromagnetic forces on atomic electrons, imparting energy.
% In single electronic collisions, only a small fraction of energy is transferred; deflection is also small.
% Alpha particles therefore travel nearly straight, continuously losing energy in many collisions.

% The average linear rate of energy loss (stopping power) for a heavy charged particle
% is given by the Bethe formula:
\begin{equation}
    -\frac{dE}{dx} = \frac{4\pi k_0^2 z^2 e^4 n}{m c^2 \beta^2}
    \left[ \ln \frac{2 m c^2 \beta^2}{I(1-\beta^2)} - \beta^2 \right]
    \label{eq:bethe}
\end{equation}
% where:
%   k_0 = 8.99e9 N m^2 C^{-2}
%   z = atomic number of projectile (z=2 for alpha)
%   e = electron charge
%   n = number of electrons per unit volume in medium
%   m = electron rest mass
%   beta = v/c of the particle
%   I = mean excitation energy of the medium

% The mean excitation energy I depends on atomic number z_m of the medium:
%   I ~ 19.0 eV                  for z_m = 1
%   I ~ 11.2 + 11.7 * z_m  eV   for 2 <= z_m <= 13
%   I ~ 52.8 + 8.71 * z_m  eV   for z_m > 13

% For non-relativistic heavy particles (like 5.48 MeV alphas), the Bethe formula simplifies
% and predicts dE/dx ~ 1/E, i.e., stopping power increases as the particle slows down.
% This leads to a characteristic Bragg peak near end of range.

% The range R of a charged particle (distance traveled before coming to rest) can be
% related to the energy by:
\begin{equation}
    \frac{dE}{dx} = \frac{k}{E} \quad \Rightarrow \quad
    R_0 = \frac{E_0^2}{2k}
\end{equation}
% so E^2 vs x_eff should be linear if k is constant.

% For alpha particles in air at 15 deg C and 760 Torr (from Melissinos):
%   R = 0.56 * E       for E < 4 MeV
%   R = 1.24 * E - 2.62  for 4 < E < 8 MeV

% Two definitions of range:
%   - Mean range R_m: absorber thickness that reduces alpha count by half
%   - Extrapolated range R_e: extrapolate linear portion of transmission curve to zero

% The effective distance through gas at reduced pressure is:
\begin{equation}
    X_{\text{eff}} = \frac{D \cdot p}{P_{\text{atm}}}
    \label{eq:xeff}
\end{equation}
% where D = source-to-detector distance, p = chamber pressure, P_atm = atmospheric pressure.
% This allows us to vary the effective path length without physically moving the source.

% Our goal: measure the range and energy loss of 5.48 MeV alpha particles from Am-241
% in four gases: air, helium, argon, and nitrogen. We compare results to theoretical
% predictions from the Bethe formula and literature values.

%===============================================================================
\section{Procedure}
\label{sec:procedure}
%===============================================================================

\subsection{Experimental Setup}
\label{sec:setup}

Apparatus: vacuum chamber with permanently mounted solid-state detector (Ortec TU-014-050-100,
S/N 36-104FF18, ion-implanted silicon, $50 mm^2$ sensitive area, 14 keV energy resolution)
and Am-241 alpha source (5.4431, 5.4857 MeV alpha energies; $Q_alpha = 5.638$ MeV).
Source is movable via rack-and-pinion drive.

Electronics chain (see figure \ref{fig:block_diagram}):
  Am-241 source -> solid-state detector -> charge-sensitive preamplifier (Canberra, 110 MOhm
  in series with detector) -> variable attenuator (Kay) -> amplifier (Ortec Model 450)
  -> MCA (Amptek Pocket MCA-8000A, connected via COM5) and oscilloscope

Detector biased at +40V (positive bias, p-layer negative w.r.t. n-layer).
One e-h pair per 3.62 eV energy deposited in silicon at 300K.

Vacuum system: mechanical pump, digital manometer (absolute pressure in mmHg).
Chamber not pumped below 50 mmHg to prevent detector damage.
Gases available: air (ambient), helium, argon, nitrogen (supplied from external tanks).

\begin{figure}[htbp]
    \centering
    \includegraphics[width=0.8\textwidth]{figures/block_diagram.png}
    \caption{Block diagram of the electronics chain.}
    \label{fig:block_diagram}
\end{figure}

\subsection{Energy Calibration}
\label{sec:calibration}

Used a pulse generator to inject known charge into the preamplifier test input via
a precisely known capacitor $C_1 = 0.4095 +/- 0.001$ pF.

Energy equivalent of pulser voltage:
\begin{equation}
    E(\text{MeV}) = 2.26 \times 10^{13} \cdot C_1 \cdot V_{\text{pulser}}
    \label{eq:calibration}
\end{equation}

Pulser output measured under loaded conditions (93 Ohm cable + 93 Ohm termination at scope).

Pulser settings: 1 kHz frequency, 0.330V offset, 5 microsec width, 5 ns edge.
Scope: Ch1 vertical 0.200 V, trigger 0.3 V; Ch2 vertical 2V, time scale 4 microsec.

MCA settings for all measurements:
  - 3 dB attenuation
  - 1024 ADC Gain
  - 10V maximum input voltage
  - 40V detector bias

Collected calibration spectra at 12 pulser voltages (15 s each), ranging from
0.035 V (0.32 MeV equivalent) to 0.575 V (5.32 MeV equivalent).
Determined centroid channel for each peak.

Calibration data:

% | Pulser Voltage (V) | Equiv. Energy (MeV) | Centroid Channel | Std (channels) |
% |---------------------|---------------------|------------------|----------------|
% | 0.035               | 0.324               | 59.04            | 0.85           |
% | 0.075               | 0.694               | 124.18           | 0.85           |
% | 0.125               | 1.157               | 208.89           | 1.59           |
% | 0.175               | 1.620               | 292.37           | 2.80           |
% | 0.225               | 2.082               | 375.56           | 1.62           |
% | 0.275               | 2.545               | 453.39           | 1.60           |
% | 0.325               | 3.008               | 542.05           | 4.30           |
% | 0.375               | 3.471               | 618.12           | 3.34           |
% | 0.425               | 3.933               | 708.89           | 1.60           |
% | 0.475               | 4.396               | 783.09           | 0.85           |
% | 0.525               | 4.859               | 875.60           | 6.20           |
% | 0.575               | 5.321               | 953.50           | 0.87           |

\begin{table}[htbp]
    \centering
    \caption{Energy calibration data}
    \label{tab:calibration-data}
    \begin{tabular}{cccc}
        \toprule
        Pulser Voltage (V) & Equiv. Energy (MeV) & Centroid Channel & Std (channels) \\
        \midrule
        0.035 & 0.324 & 59.04  & 0.85 \\
        0.075 & 0.694 & 124.18 & 0.85 \\
        0.125 & 1.157 & 208.89 & 1.59 \\
        0.175 & 1.620 & 292.37 & 2.80 \\
        0.225 & 2.082 & 375.56 & 1.62 \\
        0.275 & 2.545 & 453.39 & 1.60 \\
        0.325 & 3.008 & 542.05 & 4.30 \\
        0.375 & 3.471 & 618.12 & 3.34 \\
        0.425 & 3.933 & 708.89 & 1.60 \\
        0.475 & 4.396 & 783.09 & 0.85 \\
        0.525 & 4.859 & 875.60 & 6.20 \\
        0.575 & 5.321 & 953.50 & 0.87 \\
        \bottomrule
    \end{tabular}
\end{table}

A linear fit $E = a + b \cdot \text{ch}$ to the twelve calibration points yields
\begin{equation}
    a = -0.00568 \pm 0.02777\ \text{MeV}, \quad
    b = 0.00559 \pm 4.97 \times 10^{-5}\ \text{MeV/channel},
    \label{eq:calib-fit}
\end{equation}
with $R^2 = 0.9997$, confirming excellent linearity over the full 0.3--5.3\,MeV range.
The intercept is consistent with zero within its uncertainty, indicating no measurable
offset in the electronics chain. This fit is used in all subsequent energy calculations.

\begin{figure}[htbp]
    \centering
    \includegraphics[width=0.8\textwidth]{plots/Images/calibration-plot.png}
    \caption{Energy calibration: MCA channel number vs.\ equivalent energy (MeV).
    Linear fit with $R^2 = 0.9997$.}
    \label{fig:calibration}
\end{figure}

\subsection{Measurement Procedure}
\label{sec:measurement}

In air at atmospheric pressure, we moved the source away from the detector until counts disappeared completely.
The source disappeared at 108 mm on the scale (distance from detector) (at trigger 100 mV/div, STP).
Locked source at this position (108 mm) for all subsequent measurements.

Source-to-detector distance D: determined from the 108 mm scale reading and the
chamber geometry (see figure \ref{fig:source-detector}). Need to compute actual separation.

\begin{figure}[htbp]
    \centering
    \includegraphics[width=0.6\textwidth]{figures/source-detector.png}
    \caption{Source-detector distance.}
    \label{fig:source-detector}
\end{figure}

For each gas, varied chamber pressure and recorded MCA spectra.
Collection time: 45 s per spectrum (initially tried 15 s but data was too noisy).
Pressure measured with digital manometer, accuracy +/- 0.1 mmHg.

After collecting each spectrum, recorded:
  - Chamber pressure (mmHg)
  - Approximate peak channel (centroid)
  - Total counts detected

Converted channel -> energy using the calibration fit.
Computed $X_eff = D * p / P_atm$ for each pressure point.
Computed $E^2$ for $E^2$ vs $X_eff$ analysis.

Pressure uncertainties: standard deviation of pressure during each 45 s acquisition
was recorded for each trial.

\subsubsection{Air}
\label{sec:air_procedure}

Measured at 16 pressures from 54.8 to 765.1 mmHg.
At 765.1 mmHg: no counts detected (beyond range).
At 721.2 mmHg: barely visible, channel ~56, only 54 total counts.
At 674.9 mmHg and below: clear peaks with ~1000--1200 counts per 45 s.
Took additional high-pressure points (686.6, 700.8, 711.9 mmHg) for better
resolution near the cutoff where counts drop off.

Also collected distance-variation data at atmospheric pressure:
source moved from 94 mm to 106 mm in 2 mm steps (8 points).

\subsubsection{Helium}
\label{sec:he_procedure}

Flushed chamber with helium. Some initial issues with vacuum controls after flushing,
but self-corrected after a few trials.

Measured at 19 pressures from 50.9 to 2000+ mmHg.
Helium has much longer range than air -- needed to go to pressures well above
atmospheric (up to ~2000 mmHg) to see significant energy loss.
At 50.9 mmHg: observed clipping in MCA; this data point is suspect.
Even at 2000+ mmHg, peak was still at channel ~624 (significant remaining energy).

\subsubsection{Argon}
\label{sec:ar_procedure}

Measured at 16 pressures from 50.4 to 782.0 mmHg.
At 782.0 mmHg: only 1 count detected (beyond range).
At 770.8 mmHg: channel ~50, only 40 counts -- near the cutoff.
Observed clipping at around channel 47 (affects 770.8 and 759.6 mmHg data).
At 726.4 mmHg: 1987 total counts (unusually high -- collected for longer or
pressure was stable at a point with high transmission).

An initial argon run at $s = 10.8$\,cm (same as air) was discarded after
reviewing the lab documentation; data were re-collected with the source at
the range-out position $s = 11.1$\,cm ($D = 4.52 \pm 0.13$\,cm).

\subsubsection{Nitrogen}
\label{sec:n2_procedure}

Measured at 24 pressures from 50.8 to 782.3 mmHg.
At 782.3 mmHg: only 1 count (beyond range).
At 749.4 mmHg: 2 counts. At 727.8 mmHg: 17 counts. Gradual onset.

Took many additional points around 250--440 mmHg and 650--730 mmHg for
better resolution in regions of interest.

NOTE: Possible outliers at ~420 and ~256 Torr. Tested many other pressures
around these and concluded these points were likely affected by outside variables.
Nitrogen range is similar to air (expected, since air is ~78\% N2).

%===============================================================================
\section{Data Analysis and Results}
\label{sec:results}
%===============================================================================

\subsection{Energy Calibration}
\label{sec:results-calibration}

The energy calibration (described in \S\ref{sec:calibration}) establishes the
linear relationship between MCA channel number and particle energy.
The fit parameters from Eq.~\ref{eq:calib-fit} are obtained by a weighted linear
regression to the twelve pulser-voltage data points shown in
Table~\ref{tab:calibration-data} and Fig.~\ref{fig:calibration}.
The slope $b = 0.00559$\,MeV/channel and near-zero intercept confirm
that the system is well-behaved and no gain correction is required.

\subsection{Count Rate vs.\ Effective Distance}
\label{sec:countrate}

For each gas, the count rate (counts per second) was plotted as a function of
effective distance $X_\text{eff} = Dp/P_\text{atm}$, where $D$ is the
source-to-detector distance and $p$ is the chamber pressure.
The source-to-detector distances used were $D = 4.22 \pm 0.13$\,cm for air
and nitrogen, and $D = 4.52 \pm 0.13$\,cm for argon.

For air, argon, and nitrogen, the count rate is approximately constant at low
$X_\text{eff}$ (low pressure), then drops sharply as $X_\text{eff}$ approaches
the particle range. This is characteristic behavior: alphas lose nearly all
their energy near the end of their path (Bragg peak), so a small increase
in effective path length near the range causes a large reduction in detected
count rate.

% [Tyler]: Describe the ALFA1 fit model used in Origin:
%   Equation: n0*(1 - integral(intgauss, 0, p, a, p0))
%   What is p0 physically? (mean range) What is 'a'? (related to straggling width?)
%   I think this section should explain the physical meaning of each fit parameter.

The count rate data were fit in OriginPro using a user-defined model (ALFA1)
based on the integral of a Gaussian, which accounts for range straggling.
% [Tyler]: Expand on the model derivation/justification.
The fit parameters for each gas are listed in Table~\ref{tab:countrate-fits}.

\begin{table}[htbp]
    \centering
    \caption{Count rate fit parameters (ALFA1 model) for air, argon, and nitrogen.
             $n_0$ is the plateau count rate, $p_0$ is the mean range, and $a$
             is a straggling parameter. % [Tyler]: confirm parameter definitions.
             Helium showed no drop in count rate over the measured range
             and is excluded from this table.}
    \label{tab:countrate-fits}
    \begin{tabular}{lccc}
        \toprule
        Gas & $n_0$ (s$^{-1}$) & $a$ & $p_0$ (cm) \\
        \midrule
        Air      & $23.83 \pm 0.33$ & $0.0919 \pm 0.0081$ & $3.925 \pm 0.004$ \\
        Argon    & $23.37 \pm 0.52$ & $0.0700 \pm 0.0095$ & $4.195 \pm 0.005$ \\
        Nitrogen & $23.28 \pm 0.27$ & $0.0870 \pm 0.0066$ & $3.943 \pm 0.003$ \\
        \bottomrule
    \end{tabular}
\end{table}

Helium was also measured, but no range-out was observed over the full range of
accessible pressures (up to approximately 2000\,mmHg, corresponding to
$X_\text{eff} \approx 11$\,cm). The count rate for helium showed no
statistically significant trend with $X_\text{eff}$ (linear fit slope
$0.200 \pm 0.099$\,s$^{-1}$\,cm$^{-1}$, consistent with zero). Helium is
therefore excluded from the direct range determination and is treated separately
via energy analysis only.

% TODO: Discuss peak broadening with increasing pressure -- why do the
% MCA peaks get wider as pressure increases? Answer?: more scattering collisions
% -> more energy straggling -> broader energy distribution at the detector.

\subsection{Energy Squared vs.\ Effective Distance}
\label{sec:e2}

If the energy loss per unit distance follows $dE/dx = k/E$ (as expected for heavy
charged particles in the relevant energy regime), then integrating from $E_0$ to zero
over the full range $R_0$ gives the range:
\begin{equation}
    R_0 = \frac{E_0^2}{2k},
    \label{eq:range}
\end{equation}
and the energy as a function of position:
\begin{equation}
    E^2 = E_0^2 - 2k \cdot X_\text{eff}.
    \label{eq:e2-linear}
\end{equation}
Thus $E^2$ should be a linear decreasing function of $X_\text{eff}$ with slope $-2k$
and $y$-intercept $E_0^2$. The range $R_0$ is then obtained as $R_0 = E_0^2 / (2k)$,
equivalently the $x$-intercept of the $E^2$ vs.\ $X_\text{eff}$ line.

$E^2$ vs.\ $X_\text{eff}$ was plotted for all four gases and fit with a linear
model (Figs.~\ref{fig:air-e2}--\ref{fig:he-e2}). % [Tyler]: verify figure labels match
The resulting fit parameters are given in Table~\ref{tab:e2-fits}.

\begin{table}[htbp]
    \centering
    \caption{Linear fit parameters for $E^2$ vs.\ $X_\text{eff}$.
             The slope equals $-2k$ and the $x$-intercept gives the range $R_0$.
             Helium values are extrapolated (range not directly observed).}
    \label{tab:e2-fits}
    \begin{tabular}{lcccc}
        \toprule
        Gas & $E_0^2$ (MeV$^2$) & Slope (MeV$^2$/cm) & $k$ (MeV$^2$/cm) & $R_0$ (cm) \\
        \midrule
        Air      & $29.95 \pm 0.45$ & $-8.473 \pm 0.227$ & $4.237 \pm 0.114$ & $3.535 \pm 0.109$ \\
        Argon    & $29.57 \pm 0.49$ & $-7.917 \pm 0.242$ & $3.959 \pm 0.121$ & $3.734 \pm 0.130$ \\
        Nitrogen & $29.68 \pm 0.44$ & $-8.368 \pm 0.228$ & $4.184 \pm 0.114$ & $3.547 \pm 0.110$ \\
        Helium   & $30.77 \pm 0.28$ & $-1.649 \pm 0.058$ & $0.825 \pm 0.029$ & $18.65 \pm 0.68$ \\
        \bottomrule
    \end{tabular}
\end{table}

% R_0 = intercept / |slope|, uncertainty via R_0 * sqrt((sigma_a/a)^2 + (sigma_b/b)^2)
% Air:  R_0 = 29.951/8.473 = 3.536 cm,  sigma = 3.536*sqrt((0.446/29.95)^2+(0.227/8.473)^2) = 0.108 cm
% Ar:   R_0 = 29.566/7.917 = 3.735 cm,  sigma = 3.735*sqrt((0.485/29.57)^2+(0.242/7.917)^2) = 0.130 cm
% N2:   R_0 = 29.678/8.368 = 3.547 cm,  sigma = 3.547*sqrt((0.443/29.68)^2+(0.228/8.368)^2) = 0.110 cm
% He:   R_0 = 30.766/1.649 = 18.66 cm,  sigma = 18.66*sqrt((0.282/30.77)^2+(0.058/1.649)^2) = 0.677 cm

The $E^2$ vs.\ $X_\text{eff}$ relationship is linear for all gases in the
well-measured energy region ($E \gtrsim 1$\,MeV). Data points at very low energies
(near the end of range) deviate from the linear fit; these correspond to
conditions where the alpha barely reaches the detector and the MCA peak channel
is less reliable.
% [Tyler]: Confirm whether those near-zero-energy points were included or
% excluded from the linear fit, and note any zero-count replacement used.

The intercept $E_0^2$ can be compared to the known Am-241 alpha energy:
$E_0 = 5.486$\,MeV gives $E_0^2 = 30.10$\,MeV$^2$. All four gases yield
intercepts consistent with this value within $1$--$2\sigma$, providing an
independent consistency check on the calibration.

% Sanity check: from air E vs Xeff quadratic fit, intercept = 5.485 +/- 0.031 MeV
% (extracted directly from Origin). Known Am-241 principal energy = 5.4857 MeV.
% Agreement is excellent.

\begin{figure}[htbp]
    \centering
    \includegraphics[width=0.85\textwidth]{plots/Images/air_count_rate.png}
    \caption{Count rate vs.\ $X_\text{eff}$ for alpha particles in air.
             The plateau at $\approx 24$\,s$^{-1}$ is followed by a sharp drop
             near the mean range $p_0 = 3.925$\,cm. The red curve is the ALFA1 fit.}
    \label{fig:air-countrate}
\end{figure}

\begin{figure}[htbp]
    \centering
    \includegraphics[width=0.85\textwidth]{plots/Images/air_e2.png}
    \caption{$E^2$ vs.\ $X_\text{eff}$ for alpha particles in air, with linear fit.
             Slope $= -8.473 \pm 0.227$\,MeV$^2$/cm, giving $k = 4.237$\,MeV$^2$/cm
             and extrapolated range $R_0 = 3.535 \pm 0.109$\,cm.}
    \label{fig:air-e2}
\end{figure}

\begin{figure}[htbp]
    \centering
    \includegraphics[width=0.85\textwidth]{plots/Images/ar_countrate.png}
    \caption{Count rate vs.\ $X_\text{eff}$ for alpha particles in argon.
             Mean range $p_0 = 4.195$\,cm from ALFA1 fit.}
    \label{fig:ar-countrate}
\end{figure}

\begin{figure}[htbp]
    \centering
    \includegraphics[width=0.85\textwidth]{plots/Images/ar_e2.png}
    \caption{$E^2$ vs.\ $X_\text{eff}$ for argon.
             Slope $= -7.917 \pm 0.242$\,MeV$^2$/cm, $R_0 = 3.734 \pm 0.130$\,cm.}
    \label{fig:ar-e2}
\end{figure}

\begin{figure}[htbp]
    \centering
    \includegraphics[width=0.85\textwidth]{plots/Images/n2_count.png}
    \caption{Count rate vs.\ $X_\text{eff}$ for alpha particles in nitrogen.
             Mean range $p_0 = 3.943$\,cm. The outlier near $X_\text{eff} \approx 2$\,cm
             % [Tyler]: explain this outlier -- was it excluded from the fit?
             lies well below the plateau.}
    \label{fig:n2-countrate}
\end{figure}

\begin{figure}[htbp]
    \centering
    \includegraphics[width=0.85\textwidth]{plots/Images/n2_e2.png}
    \caption{$E^2$ vs.\ $X_\text{eff}$ for nitrogen.
             Slope $= -8.368 \pm 0.228$\,MeV$^2$/cm, $R_0 = 3.547 \pm 0.110$\,cm.}
    \label{fig:n2-e2}
\end{figure}

\begin{figure}[htbp]
    \centering
    \includegraphics[width=0.85\textwidth]{plots/Images/He_count.png}
    \caption{Count rate vs.\ $X_\text{eff}$ for helium. No drop-off is observed;
             the linear fit slope ($0.20 \pm 0.10$\,s$^{-1}$\,cm$^{-1}$) is
             consistent with zero. The alpha range in helium exceeds the maximum
             achievable $X_\text{eff}$ at the pressures measured.}
    \label{fig:he-countrate}
\end{figure}

\begin{figure}[htbp]
    \centering
    \includegraphics[width=0.85\textwidth]{plots/Images/he_e2.png}
    \caption{$E^2$ vs.\ $X_\text{eff}$ for helium. Linear fit over the measured
             range gives slope $= -1.649 \pm 0.058$\,MeV$^2$/cm,
             extrapolating to $R_0 = 18.65 \pm 0.68$\,cm.}
    \label{fig:he-e2}
\end{figure}

%===============================================================================
\section{Discussion and Conclusions}
\label{sec:conclusion}
%===============================================================================

% TODO: Discuss results in context.
%
% - How well does E^2 vs X_eff linearity hold? Where does it break down?
% - Meaning of k varying with energy (Bethe formula predicts log dependence, not pure 1/E)
% - Comparison of range in different gases to theoretical expectations
% - Discuss straggling, Bragg peak
% - Sources of systematic error (clipping at low channels, vacuum control issues with He)
% - Comparison to example report's ~10% discrepancy with literature

\nocite{*}
\bibliographystyle{alpha}
\bibliography{refs}
\end{document}
