\documentclass{article}[11pt]

\usepackage{arxiv}

\usepackage[utf8]{inputenc}
\usepackage[T1]{fontenc}
\usepackage{hyperref}
\usepackage{url}
\usepackage{booktabs}
\usepackage{amsfonts}
\usepackage{amsmath}
\usepackage{amssymb}
\usepackage{nicefrac}
\usepackage{microtype}
\usepackage{cleveref}
\usepackage{graphicx}
\usepackage[numbers,sort&compress]{natbib}
\usepackage{doi}
\usepackage{siunitx}
\usepackage{setspace}
\usepackage{url}

%Macros
\newcommand{\paren}[1]{\left(#1\right)}
\newcommand{\sbrak}[1]{\left[#1\right]}
\newcommand{\cbrak}[1]{\left\{#1\right\}}

\newcommand{\bra}[1]{\left\langle #1 \right\rvert}
\newcommand{\ket}[1]{\left\lvert #1 \right\rangle}
\newcommand{\braket}[2]{\left\langle #1 \left.\right\lvert #2 \right\rangle}
\newcommand{\ketbra}[2]{\left\lvert #1 \right\rangle\!\!\left\langle #2 \right\lvert}

\newcommand{\jand}{\quad\text{and}\quad}
\newcommand{\jor}{\quad\text{or}\quad}
\newcommand{\for}{\quad\text{for }\,}
\newcommand{\comma}{,\quad}

\newcommand{\ieval}[2]{\Bigg|^{#2}_{#1}}
\newcommand{\deval}[1]{\bigg|_{#1}}

\newcommand{\diff}[2]{\frac{d#1}{d#2}}
\newcommand{\pdiff}[2]{\frac{\partial#1}{\partial#2}}
%End Macros

\onehalfspacing


\title{\textbf{Range of Alpha Particles in Various Gas}}

\date{\today}

\author{Aidan Andrews\\
%   Department of Physics\\
%   University of Illinois at Urbana-Champaign\\
  \texttt{aidansa2}\\
  \And
  Tyler Wang\\
%   Department of Physics\\
%   University of Illinois at Urbana-Champaign\\
  \texttt{tylerww3}\\
}

\renewcommand{\headeright}{}
\renewcommand{\undertitle}{University of Illinois at Urbana-Champaign \\ PHYS 403}
\renewcommand{\shorttitle}{Range of Alpha Particles in Gas}

\hypersetup{
	pdftitle={Range of Alpha Particles in Gas},
	pdfsubject={Nuclear Physics, Alpha Particles, Energy Loss},
	pdfauthor={Aidan Andrews, Tyler Wang},
	pdfkeywords={Alpha Particles, Range, Energy Loss, Bethe Formula, Am-241},
}
\allowdisplaybreaks

\begin{document}
\maketitle

\begin{abstract}
    In this experiment, we investigate the range of alpha particles from Americium 141 in various gases.
    Using a solid state detector, and by varying the pressure to simulate various distances, we obtain a relationship between energy and distance.
    With our data, we were also able to obtain a relationship between count rate and distance.
    Using this information, we were able to verify various theoretical predictions of energy loss within these media.
    We then compare these results with accepted values from the NIST.
\end{abstract}

% \keywords{Alpha Particles}

%===============================================================================
\section{Introduction}
\label{sec:intro}
%===============================================================================

As alpha particles travel through a medium, they lose energy primarily through ionization, or the excitation of electrons in surrounding atoms.
Due to the relatively low mass of alpha particles, in general, these interactions preserve the particles' original trajectories; for this reason, they travel nearly straight as they dissipate their energy.

A model for the stopping power of a medium can be found in \cite[pg. 159, eq. 2.10]{mel66}, which we restate here for convenience.
\begin{equation} \label{eq:berte}
	-\diff{E}{x}= \paren{\frac{1}{4\pi \epsilon_0}}\frac{4\pi z^2 e^4}{m_ev^2}n_e \sbrak{\ln\paren{\frac{2m_e v^2}{I(1-\beta^2)}} - \beta^2}
\end{equation}
where
\begin{align*}
	e &= \text{charge on an electron (C)} \\
	z &= \text{atomic number of moving particle (z=2 for alpha particles)} \\
	N &= \text{number of atoms per unit volumn } m^{-1} \\
	m_e &= \text{mass of an electron (kg)} \\
	v &= \text{velocity of moving particle (m/s)} \\
	E &= \text{kinetic energy of moving particle (J)} \\
	I &= \text{Mean ionization potential of medium (J)} \\
	x &= \text{traveled distance of moving particle (m)} \\
	\epsilon_0 &= \text{permativity of free space } (Nm^2/C)
\end{align*}
In this experiment, since our particles are relatively low-energy (around 5.48 MeV), we take the non-relativistic limit, which \cite[pg. 158]{mel66} gives as
\begin{equation} \label{eq:berte-non-rel}
	-\diff{E}{x} \propto \frac{z^2M}{E}
\end{equation}
For some constant of proportionality, this relation can be integrated to achieve the following:
\begin{align}
	\int_{E_0}^{E(r)} E \,dE &= \int_0^{r}-kz^2M \,dx \\
	{E}^2 - {E_0}^2 &= kz^2Mr. \label{eq:e2-linear-rel}
\end{align}
Therefore, we expect a linear relationship between $E^2$: the energy of the particle, and $r$, the distance traveled, where $E_0$ is the initial energy of the particle.

In this experiment, our goal is to verify this relation by observing this linear relationship using alpha particles emitted from Americium-241.
In addition, using our data, we'd like to determine the range of these particles.

We model our procedure after a similar experiment as outlined in \cite[Ch. 5; Sec 3.2]{mel66}.
With this setup, instead of manipulating the distance directly, the distance between the sample and the detector was fixed, and the pressure within the experimental vessel was varied to simulate different distances.
To obtain the corresponding effective pressure at STP, we appeal to the following formula
\begin{equation} \label{eq:eff_distance}
	X_\text{eff} = \frac{Dp}{p_\text{atm}}
\end{equation}
where
\begin{align*}
	X_\text{eff} &= \text{effective distance at STP (m)} \\
	D &= \text{physical distance between sample and detector} \\
	p &= \text{pressure of experimentation vessel} \\
	p_\text{atm} &= \text{standard atmospheric pressure (the value 760 mmHg was used in this experiment),}
\end{align*}
as it appears in \cite{mel66}.

In addition to collecting energy information, we also seek to investigate the count rate as it varies with respect to distance.
Due to \textit{range straggling} \cite{andrews08}, we expect some slight statistical fluctuations in the range of alpha particles.
In accordance with \cite{andrews08}, the probability distribution of the range of particles can be estimated to be a Gaussian, which is given as
\begin{equation}
	P(r) = \frac{1}{\alpha \sqrt{\pi}}\exp\paren{-\frac{(x-\bar x)^2}{\alpha^2}}.
\end{equation}
where
\begin{align*}
	x &= \text{range of alpha particle (m)} \\
	\bar x &= \text{average range of alpha particle (m)} \\
	\alpha &= \text{straggling parameter}.
\end{align*}
Therefore, the following integral
\begin{equation} \label{eq:err_func}
	n_0 \paren{1 - \int_0^x	\frac{1}{\alpha \sqrt{\pi}}\exp\paren{-\frac{(x-\bar x)^2}{\alpha^2}}\,dx},
\end{equation}
denotes the expected relation between counts and distance, given some average range and some straggling parameter, where we take $n_0$ as the total number of counts emitted by the sample (per unit time).

In addition to performing this procedure in air, we repeat this experiment in nitrogen, argon, and helium atmospheres to compare the results of changing the atmosphere.


\section{Procedure}
\label{sec:procedure}
%===============================================================================

\subsection{Experimental Setup}
\label{sec:setup}

The core of the apparatus was a solid-state silicon detector (Ortec TU-014-050-100, $50\,\text{mm}^2$ active area, 14\,keV energy resolution) mounted inside a sealed vacuum chamber, with an Am-241 source on a rack-and-pinion drive that allowed fine adjustment of the source-to-detector distance. When an alpha particle enters the detector, it ionizes the silicon and produces electron-hole pairs at a rate of one pair per 3.62\,eV deposited---so the total collected charge is directly proportional to the particle's kinetic energy. That charge pulse was amplified and shaped before reaching the MCA (see Fig.~\ref{fig:block_diagram}), which sorted pulses into 1024 channels by amplitude; the resulting histogram is the alpha energy spectrum.

Rather than mechanically varying the source-to-detector distance for each data point, we controlled the gas pressure inside the chamber to change the effective path length. A mechanical pump evacuated the chamber down to $\sim$50\,mmHg (below which the detector risks radiation damage), and the chamber could be backfilled with helium, argon, or nitrogen from external cylinders in addition to ambient air.

\begin{figure}[htbp]
    \centering
    \includegraphics[width=0.8\textwidth]{figures/block_diagram.png}
    \caption{Block diagram of the electronics chain.}
    \label{fig:block_diagram}
\end{figure}

\subsection{Energy Calibration}
\label{sec:calibration}


Used a pulse generator to inject known charge into the preamplifier test input via
a precisely known capacitor $C_1 = 0.4095 +/- 0.001$ pF.

Energy equivalent of pulser voltage:
\begin{equation}
    E(\text{MeV}) = 2.26 \times 10^{13} \cdot C_1 \cdot V_{\text{pulser}}
    \label{eq:calibration}
\end{equation}

Pulser output measured under loaded conditions (93 Ohm cable + 93 Ohm termination at scope).

Pulser settings: 1 kHz frequency, 0.330V offset, 5 $\mu$s width, 5 ns edge.
Scope: Ch1 vertical 0.200 V, trigger 0.3 V; Ch2 vertical 2V, time scale 4 $\mu$s.

MCA settings for all measurements:
  - 3 dB attenuation
  - 1024 ADC Gain
  - 10V maximum input voltage
  - The amplifier bias for the detector was set at 40V

Collected calibration spectra at 12 pulser voltages (15 s each), ranging from
0.035 V (0.32 MeV equivalent) to 0.575 V (5.32 MeV equivalent).
Determined centroid channel for each peak.

Calibration data:

% | Pulser Voltage (V) | Equiv. Energy (MeV) | Centroid Channel | Std (channels) |
% |---------------------|---------------------|------------------|----------------|
% | 0.035               | 0.324               | 59.04            | 0.85           |
% | 0.075               | 0.694               | 124.18           | 0.85           |
% | 0.125               | 1.157               | 208.89           | 1.59           |
% | 0.175               | 1.620               | 292.37           | 2.80           |
% | 0.225               | 2.082               | 375.56           | 1.62           |
% | 0.275               | 2.545               | 453.39           | 1.60           |
% | 0.325               | 3.008               | 542.05           | 4.30           |
% | 0.375               | 3.471               | 618.12           | 3.34           |
% | 0.425               | 3.933               | 708.89           | 1.60           |
% | 0.475               | 4.396               | 783.09           | 0.85           |
% | 0.525               | 4.859               | 875.60           | 6.20           |
% | 0.575               | 5.321               | 953.50           | 0.87           |

\begin{table}[htbp]
    \centering
    \caption{Energy calibration data}
    \label{tab:calibration-data}
    \begin{tabular}{cccc}
        \toprule
        Pulser Voltage (V) & Equiv. Energy (MeV) & Centroid Channel & Std (channels) \\
        \midrule
        0.035 & 0.324 & 59.04  & 0.85 \\
        0.075 & 0.694 & 124.18 & 0.85 \\
        0.125 & 1.157 & 208.89 & 1.59 \\
        0.175 & 1.620 & 292.37 & 2.80 \\
        0.225 & 2.082 & 375.56 & 1.62 \\
        0.275 & 2.545 & 453.39 & 1.60 \\
        0.325 & 3.008 & 542.05 & 4.30 \\
        0.375 & 3.471 & 618.12 & 3.34 \\
        0.425 & 3.933 & 708.89 & 1.60 \\
        0.475 & 4.396 & 783.09 & 0.85 \\
        0.525 & 4.859 & 875.60 & 6.20 \\
        0.575 & 5.321 & 953.50 & 0.87 \\
        \bottomrule
    \end{tabular}
\end{table}

A linear fit $E = a + b \cdot \text{ch}$ to the twelve calibration points yields
\begin{equation}
    a = -0.00568 \pm 0.02777\ \text{MeV}, \quad
    b = 0.00559 \pm 4.97 \times 10^{-5}\ \text{MeV/channel},
    \label{eq:calib-fit}
\end{equation}
with $R^2 = 0.9997$, confirming excellent linearity over the full 0.3--5.3\,MeV range.
The intercept is consistent with zero within its uncertainty, indicating no measurable
offset in the electronics chain. This fit is used in all subsequent energy calculations.

% Include calibration plot figure here
\begin{figure}[htbp]
    \centering
    \includegraphics[width=0.8\textwidth]{plots/Images/calibration-plot.png}
    \caption{Energy calibration: MCA channel number vs.\ equivalent energy (MeV).
    Linear fit with $R^2 = 0.9997$.}
    \label{fig:calibration}
\end{figure}

\subsection{Measurement Procedure}
\label{sec:measurement}

To keep the source-to-detector distance fixed across all measurements, we used the ``range-out'' position---the point at which alpha particles can no longer reach the detector at atmospheric pressure. We found this by slowly moving the source away from the detector until counts dropped to zero, which happened at a scale reading of $s = 10.8$\,cm. The source was locked there for all four gases. The physical source-to-detector distance follows from the chamber geometry as $D = (s - 6.58)\,\text{cm} = 4.22 \pm 0.13$\,cm.

With the source fixed, we swept the chamber pressure from $\sim$50\,mmHg up to near atmospheric and collected a 45-s MCA spectrum at each pressure point. We initially tried 15-s runs but the statistics were too poor to reliably locate the peak centroid, so all reported data use 45\,s. At each pressure we pulled the centroid channel from the spectrum and converted it to energy using the calibration from \S\ref{sec:calibration}. The effective path length at STP is then $X_\text{eff} = Dp/P_\text{atm}$, where $p$ is the chamber pressure and $P_\text{atm}$ is atmospheric pressure on the day of the measurement. Pressure readings carry an uncertainty of $\pm 0.1$\,mmHg.

\subsubsection{Air}
\label{sec:air_procedure}

We took 16 spectra in air at pressures from 54.8 to 765.1\,mmHg. At full atmospheric pressure no counts arrived at all, and at 721.2\,mmHg only 54 counts appeared over 45\,s---right at the edge of the range. Below 674.9\,mmHg the count rate settled into a stable plateau of roughly 1000--1200 counts per 45\,s. Because the count rate drops sharply near range-out, we added three extra points at 686.6, 700.8, and 711.9\,mmHg to better resolve that transition region.

\subsubsection{Helium}
\label{sec:he_procedure}

Helium has a much lower electron density than the diatomic gases, so alpha particles lose energy more slowly per centimeter and the range is much longer. We needed to go well above atmospheric pressure to see significant energy loss over our fixed distance, so we took 19 spectra from 50.9\,mmHg up to $\sim$2000\,mmHg. Even at the highest pressures the MCA peak only shifted down to around channel 624, meaning the particles were still arriving with substantial energy---we never observed range-out in helium. The spectrum at 50.9\,mmHg showed MCA clipping and was dropped from the analysis.

\subsubsection{Argon}
\label{sec:ar_procedure}

In argon, we took 16 spectra from 50.4 to 782.0\,mmHg, with the source at the same $s = 10.8$\,cm position used for all other gases. Near the cutoff, at 770.8 and 759.6\,mmHg, the peak landed close to channel 47 where MCA clipping occurs, so those two points were excluded from the fits. At 782.0\,mmHg only a single count appeared over 45\,s, confirming the particles had essentially stopped.

\subsubsection{Nitrogen}
\label{sec:n2_procedure}

We collected 24 spectra in nitrogen from 50.8 to 782.3\,mmHg. Since air is roughly 78\% nitrogen, we expected similar behavior to air, and the range-out did fall at nearly the same pressure. To get good coverage of both the drop-off region and the plateau, we added extra points in the 250--440\,mmHg and 650--730\,mmHg ranges. Two data points---near 420 and 256\,Torr---were clear outliers relative to the neighboring measurements. We took several additional spectra at nearby pressures to confirm they were anomalous rather than a real feature, and excluded them from the fits.

%===============================================================================
\section{Data Analysis and Results}
\label{sec:results}
%===============================================================================

\subsection{Energy Calibration}
\label{sec:results-calibration}
The data collected during the energy calibration procedure (described in \S\ref{sec:calibration}) yields a linear relationship (shown in
Table~\ref{tab:calibration-data} and Fig.~\ref{fig:calibration}), which is what we expect given the specifications of the equipment provided by the manufacturer.
Performing a linear regression, we obtain the formula
\begin{align}
	E &= (5.59*10^{-3})C - 0.01 \\
	\sigma_E &= \sqrt{(0.03)^2 + ((5\times 10^{-5})C)^2 + ((5.59*10^{-3})\sigma_C)^2}
\end{align}

\subsection{Alpha Particle Range}
\label{sec:alpharange}
\subsubsection{Air}
The first task was to convert pressure to effective distance with equation \ref{eq:eff_distance}.
The distance used for this measurement was $4.22 \pm 0.13$  cm (and a standard $760$ mmHg was used as $p_\text{atm}$).
Then uncertainty can be obtained through the following formula
\begin{equation}
	\sigma_{X_\text{eff}} = \sqrt{\paren{(p\sigma_{D}/p_\text{atm}}^2+\paren{(\sigma_{p}D/p_\text{atm}}^2}
\end{equation}

For air, plotting the count rate against the effective distance yields the graph in figure \ref{fig:air-counts}.
\begin{figure}[h] \label{fig:air-counts}
	\centering
	\resizebox{0.8\linewidth}{!}{\includegraphics{plots/PyPlots/air_counts_plot.png}}
	\caption{Count rate data plotted against effective range for experiment in air.
		Equation \ref{eq:err_func} was then used to fit the trendline}
\end{figure}
Using the data from the fit, we estimate the average range is $3.7 \pm 0.5$ cm with a straggling parameter of $0.3 \pm 0.5$.


Moving on, we plot the square of the energy against effective distance and obtain the plot in figure \ref{fig:air-energy}.
\begin{figure}[h] \label{fig:air-energy}
	\centering
	\resizebox{0.8\linewidth}{!}{\includegraphics{plots/PyPlots/air_e2.png}}
	\caption{Energy squared plotted against energy for experiment in air.
	A scaled Gaussian (using the parameters from the count rate analysis) is plotted in blue.
	The trendline is obtained from a linear regression of points with $X_\text{eff}\le 3.4$.}
\end{figure}
We wish to fit a linear regression with the points before they start to range out.
Using the analysis from the count data, we know the particles begin to range out at 3.7.
However, due to statistical variation, we wish to exclude points with $X_\text{eff} \le R_0-\alpha$ (At this range, 0.92 have yet to range out).
Using this analysis, we obtain the trendline as observed in figure \ref{fig:air-energy}
The trend of this fit supports equation \ref{eq:e2-linear-rel} as outlined in \S\ref{sec:intro}.

Using the intercept of the trendline in figure \ref{fig:air-energy}, we obtain that the initial energy of these particles is $5.43 \pm 0.03$ MeV (according to \cite{qadr17}, the accepted value for Am-147 is 5.443 MeV).
In addition, by setting the $E^2 = 0$ (the point where the particles have no energy; hence range out), we can solve for $r$,
\begin{equation}
	r=-\frac{b}{a}
\end{equation}
where $a$ and $b$ are the slope and intercept, respectively.
Here, $r$ presents the range of the particles, and using our parameters, we obtain the value $3.54\pm 0.08$ cm, which is within our uncertainty from our count rate analysis.

\subsubsection{Nitrogen}
Using the same method, the plots of count rate and energy squared against effective distance are analyzed.
The results are displayed in figures \ref{fig:n2-counts} and \ref{fig:n2-energy}.\footnote{Observing figure \ref{fig:n2-counts}, at an effective distance of 1.44 cm, a point was taken that was more than 14.6$\sigma$ from the value observed from values taken near it.
	For this reason, this point was discarded when performing the analysis.}
\begin{figure}[h] \label{fig:n2-counts}
	\centering
	\resizebox{0.8\linewidth}{!}{\includegraphics{plots/PyPlots/n2_counts_plot.png}}
	\caption{Count rate data plotted against effective range for experiment in nitrogen.
		Equation \ref{eq:err_func} was then used to fit the trendline}
\end{figure}
\begin{figure}[h] \label{fig:n2-energy}
	\centering
	\resizebox{0.8\linewidth}{!}{\includegraphics{plots/PyPlots/n2_e2.png}}
	\caption{Energy squared plotted against effective range for experiment in nitrogen.
	A scaled Gaussian (using the parameters from the count rate analysis) is plotted in blue.
	The trendline is obtained from a linear regression of points with $X_\text{eff}\le 3.3$.}
\end{figure}

The corresponding alpha range as derived from the count rate is $3.7 \pm 0.5$ cm.
From the energy data, we obtain a range of $3.56 \pm 0.06$ cm and an initial energy of $5.42 \pm 0.02$ MeV.
This value is similar to that of air, which we expect since air is mostly nitrogen.

\subsubsection{Argon}
Using the same method, the plots of count rate and energy squared against effective distance are analyzed.
The results are displayed in figures \ref{fig:n2-counts} and \ref{fig:n2-energy}.
\begin{figure}[h] \label{fig:ar-counts}
	\centering
	\resizebox{0.8\linewidth}{!}{\includegraphics{plots/PyPlots/ar_counts_plot.png}}
	\caption{Count rate data plotted against effective range for experiment in argon.
		Equation \ref{eq:err_func} was then used to fit the trendline}
\end{figure}
\begin{figure}[h] \label{fig:ar-energy}
	\centering
	\resizebox{0.8\linewidth}{!}{\includegraphics{plots/PyPlots/ar_e2.png}}
	\caption{Energy squared plotted against effective range for experiment in argon.
	A scaled Gaussian (using the parameters from the count rate analysis) is plotted in blue.
	The trendline is obtained from a linear regression of points with $X_\text{eff}\le 3.7$.}
\end{figure}

The corresponding alpha range as derived from the count rate is $4.0 \pm 0.3$ cm.
From the energy data, we obtain a range of $3.83 \pm 0.14$ cm and an initial energy of $5.34 \pm 0.06$ MeV.

Since argon has a very high ionization potential, observing the $\ln$ term in equation \ref{eq:berte}, we find that this high ionization potential lowers the magnitude of this term (since the numerator in this term is orders of magnitude larger than the denominator).
Since we assume $\beta$ is small, we see that this high ionization potential contributes to a lowering in terms of
\begin{equation}
	\sbrak{\ln\paren{\frac{2m_e v^2}{I(1-\beta^2)}} - \beta^2}
\end{equation}
Therefore, this means particles in argon lose energy more slowly.
This explanation very roughly explains the deviation from the previous results.

\subsubsection{Helium}
Observing the countrate vs effective distance plot for helium in figure \ref{fig:he-counts},
\begin{figure}[h] \label{fig:he-counts}
	\centering
	\resizebox{0.8\linewidth}{!}{\includegraphics{plots/PyPlots/he_counts_plot.png}}
	\caption{Count rate data plotted against effective range for experiment in helium.}
\end{figure}
for the points we measure, helium failed to range out.
In figure \ref{fig:he-energy}, this is made even more apparent since the points don't intersect the $x$-axis.
\begin{figure}[h] \label{fig:he-energy}
	\centering
	\resizebox{0.8\linewidth}{!}{\includegraphics{plots/PyPlots/he_e2.png}}
	\caption{Energy squared plotted against effective range for experiment in argon.
	A scaled Gaussian (using the parameters from the count rate analysis) is plotted in blue.
	The trendline is obtained from a linear regression of every point.}
\end{figure}
For this reason, we didn't filter the points when we ran the linear regression for energy squared versus effective distance.
Then, using this line, like before, we can compute the range of the particle, which we find is $18.64 \pm 0.016$ cm, and the initial energy of the particles is $5.55 \pm 0.06$ MeV.
Since helium is the lightest of all of the gases (meaning the alpha particles are able to retain more energy between collisions), and since helium is a noble gas (like argon), this result is within expectation.

%===============================================================================
\section{Discussion and Conclusions}
\label{sec:conclusion}
As observed in \S\ref{sec:results}, the values that were obtained for initial energy were very close to the accepted values for Am-241.

Then, for the ranges which we computed, comparing against the values found in \cite{nistdata},\footnote{We convert their values to compare with ours by dividing by the density of the corresponding gas.
These values were found \url{https://physics.nist.gov/cgi-bin/Star/compos.pl?refer=ap&matno=104}}
we obtain table \ref{tb:acc-values}.

\begin{table} \label{tb:acc-values}
    \centering
    \caption{Measured ranges and NIST projected ranges, values obtained from \cite{nistdata}.}
    \label{tab:calibration-data}
    \begin{tabular}{cccc}
        \toprule
        Gas & Measure Range (cm) & NIST Range $(\text{g/cm}^2)$ & Converted Range (cm) \\
        \midrule
        Air 			& $3.548 \pm 0.08$ 	& $4.93\times 10^{-3}$	& 4.09 \\
        $\text{N}^2$ 	& $3.56 \pm 0.06$ 		& $4.85\times 10^{-3}$  & 4.16 \\
        Ar 				& $3.83 \pm 0.14$ 	& $7.25\times 10^{-3}$  & 4.36 \\
        He 				& $18.64 \pm 0.016$ & $3..85\times 10^{-3}$ & 23.1 \\
        \bottomrule
    \end{tabular}
\end{table}

Unfortunately, we observe some significant deviation from the values we computed.
From what we can tell, the particles we observe seem to lose energy much faster than they should.
This can be due to numerous reasons, for example, the atmosphere we experimented in could contain some impurities which might impact our result.
In addition, we noticed the multi-channel analyzer clipping near the edges, which could impact some of the data that we obtained.

In the future, to help reduce error bars, which would improve our fit for count rate versus count rate plots, we could take data for longer periods of time.
In addition, we didn't have access to the tools to monitor the uncertainty in pressure until halfway through the experiment.
More precise monitoring of the uncertainty of pressure could help us reduce the error bars in the effective distance, since we likely overestimated the error in many of these points.

However, the results we obtained successfully verified equation \ref{eq:berte-non-rel} and theoretical predictions of the energy of alpha particles produced by Am-241 atoms.


\nocite{*}
\bibliographystyle{alpha}
\bibliography{refs}
\end{document}
