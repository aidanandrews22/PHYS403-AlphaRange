\documentclass{article}[11pt]

\usepackage{arxiv}

\usepackage[utf8]{inputenc}
\usepackage[T1]{fontenc}
\usepackage{hyperref}
\usepackage{url}
\usepackage{booktabs}
\usepackage{amsfonts}
\usepackage{amsmath}
\usepackage{amssymb}
\usepackage{nicefrac}
\usepackage{microtype}
\usepackage{cleveref}
\usepackage{graphicx}
\usepackage{natbib}
\usepackage{doi}
\usepackage{siunitx}
\usepackage{setspace}

%Macros
\newcommand{\paren}[1]{\left(#1\right)}
\newcommand{\sbrak}[1]{\left[#1\right]}
\newcommand{\cbrak}[1]{\left\{#1\right\}}

\newcommand{\bra}[1]{\left\langle #1 \right\rvert}
\newcommand{\ket}[1]{\left\lvert #1 \right\rangle}
\newcommand{\braket}[2]{\left\langle #1 \left.\right\lvert #2 \right\rangle}
\newcommand{\ketbra}[2]{\left\lvert #1 \right\rangle\!\!\left\langle #2 \right\lvert}

\newcommand{\jand}{\quad\text{and}\quad}
\newcommand{\jor}{\quad\text{or}\quad}
\newcommand{\for}{\quad\text{for }\,}
\newcommand{\comma}{,\quad}

\newcommand{\ieval}[2]{\Bigg|^{#2}_{#1}}
\newcommand{\deval}[1]{\bigg|_{#1}}

\newcommand{\diff}[2]{\frac{d#1}{d#2}}
\newcommand{\pdiff}[2]{\frac{\partial#1}{\partial#2}}
%End Macros

\onehalfspacing


\title{\textbf{Range of Alpha Particles in Various Gas}}

\date{\today}

\author{Aidan Andrews\\
%   Department of Physics\\
%   University of Illinois at Urbana-Champaign\\
  \texttt{aidansa2}\\
  \And
  Tyler Wang\\
%   Department of Physics\\
%   University of Illinois at Urbana-Champaign\\
  \texttt{tylerww3}\\
}

\renewcommand{\headeright}{}
\renewcommand{\undertitle}{University of Illinois at Urbana-Champaign \\ PHYS 403}
\renewcommand{\shorttitle}{Range of Alpha Particles in Gas}

\hypersetup{
	pdftitle={Range of Alpha Particles in Gas},
	pdfsubject={Nuclear Physics, Alpha Particles, Energy Loss},
	pdfauthor={Aidan Andrews, Tyler Wang},
	pdfkeywords={Alpha Particles, Range, Energy Loss, Bethe Formula, Am-241},
}

\begin{document}
\maketitle

\begin{abstract}
    % TODO: Write after results/discussion are complete.
    % Should summarize: measured range of Am-241 alpha particles in air, He, Ar, N2
    % using solid-state detector + MCA. Compare to Bethe formula predictions.
    % Mention agreement/discrepancy with literature values.
\end{abstract}

% \keywords{Alpha Particles}

%===============================================================================
\section{Introduction}
\label{sec:intro}
%===============================================================================

As alpha particles travel through a medium, they lose energy primarily through ionization, or the excitation of electrons in surrounding atoms.
Due to the relatively low mass of alpha particles, in general, these interactions preserve the particles' original trajectories; for this reason, they travel nearly straight as they dissipate their energy.

A model for the stopping power of a medium can be found in \cite[pg. 159, eq. 2.10]{mel66}, which we restate here for convenience.
\begin{equation}
	-\diff{E}{x}= \paren{\frac{1}{4\pi \epsilon_0}}\frac{4\pi z^2 e^4}{m_ev^2}n_e \sbrak{\ln\paren{\frac{2m_e v^2}{I(1-\beta^2) - \beta^2}}}
\end{equation}
where
\begin{align*}
	e &= \text{charge on an electron (C)} \\
	z &= \text{atomic number of moving particle (z=2 for alpha particles)} \\
	N &= \text{number of atoms per unit volumn } m^{-1} \\
	m_e &= \text{mass of an electron (kg)} \\
	v &= \text{velocity of moving particle (m/s)} \\
	E &= \text{kinetic energy of moving particle (J)} \\
	I &= \text{Mean ionization potential of medium (J)} \\
	x &= \text{traveled distance of moving particle (m)} \\
	\epsilon_0 &= \text{permativity of free space } (Nm^2/C)
\end{align*}
In this experiment, since our particles are relatively low-energy (around 5.48 MeV), we take the non-relativistic limit, which \cite[pg. 158]{mel66} gives as
\begin{equation}
	-\diff{E}{x} \propto \frac{z^2M}{E}
\end{equation}

In this experiment, our goal is to verify this relation using alpha particles emitted from Americium-241.
We model our procedure after a procedure of a similar experiment as outlined in \cite[Ch. 5; Sec 3.2]{mel66}.
With this setup, instead of manipulating the distance directly, the distance between the sample and the detector in this experiment was fixed, and the pressure within the experimentation vessel was varied to simulate various distances.
To obtain the corresponding effective pressure at STP, we appeal to the following formula
\begin{equation}
	X_\text{eff} = \frac{Dp}{p_\text{atm}}
\end{equation}
where
\begin{align*}
	X_\text{eff} &= \text{effective distance at STP (m)} \\
	D &= \text{physical distance between sample and detector} \\
	p &= \text{pressure of experimentation vessel} \\
	p_\text{atm} &= \text{standard atmospheric pressure (the value 760 mmHg was used in this experiment),}
\end{align*}
as it appears in \cite{mel66}.

In addition to collecting energy information, we also seek to investigate the count rate as it varies with respect to distance.
Since we expect some slight statistical fluctuations in the range of alpha particles, we estimate the probability distribution of the range of particles to be a Gaussian, which is given as
\begin{equation}
	P(r) = \frac{1}{\alpha \sqrt{\pi}}\exp\paren{-\frac{(x-\bar x)^2}{\alpha^2}}.
\end{equation}
where
\begin{align*}
	x &= \text{range of alpha particle (m)} \\
	\bar x &= \text{average range of alpha particle (m)} \\
	\alpha &= \text{straggling parameter}.
\end{align*}
Therefore, the following integral
\begin{equation}
	n_0 \paren{1 - \int_0^x	\frac{1}{\alpha \sqrt{\pi}}\exp\paren{-\frac{(x-\bar x)^2}{\alpha^2}}},
\end{equation}
denotes the expected relation between counts and distance, given some average range and some straggling parameter, where we take $n_0$ as the total number of counts emitted by the sample (per unit time).

In addition to performing this procedure in air, we repeat this experiment in nitrogen, argon, and helium atmospheres to compare the results of changing atmospheres.


\section{Procedure}
\label{sec:procedure}
%===============================================================================

\subsection{Experimental Setup}
\label{sec:setup}

The core of the apparatus was a solid-state silicon detector (Ortec TU-014-050-100, $50\,\text{mm}^2$ active area, 14\,keV energy resolution) mounted inside a sealed vacuum chamber, with an Am-241 source on a rack-and-pinion drive that allowed fine adjustment of the source-to-detector distance. When an alpha particle enters the detector, it ionizes the silicon and produces electron-hole pairs at a rate of one pair per 3.62\,eV deposited---so the total collected charge is directly proportional to the particle's kinetic energy. That charge pulse was amplified and shaped before reaching the MCA (see Fig.~\ref{fig:block_diagram}), which sorted pulses into 1024 channels by amplitude; the resulting histogram is the alpha energy spectrum.

Rather than mechanically varying the source-to-detector distance for each data point, we controlled the gas pressure inside the chamber to change the effective path length. A mechanical pump evacuated the chamber down to $\sim$50\,mmHg (below which the detector risks radiation damage), and the chamber could be backfilled with helium, argon, or nitrogen from external cylinders in addition to ambient air.

\begin{figure}[htbp]
    \centering
    \includegraphics[width=0.8\textwidth]{figures/block_diagram.png}
    \caption{Block diagram of the electronics chain.}
    \label{fig:block_diagram}
\end{figure}

\subsection{Energy Calibration}
\label{sec:calibration}


Used a pulse generator to inject known charge into the preamplifier test input via
a precisely known capacitor $C_1 = 0.4095 +/- 0.001$ pF.

Energy equivalent of pulser voltage:
\begin{equation}
    E(\text{MeV}) = 2.26 \times 10^{13} \cdot C_1 \cdot V_{\text{pulser}}
    \label{eq:calibration}
\end{equation}

Pulser output measured under loaded conditions (93 Ohm cable + 93 Ohm termination at scope).

Pulser settings: 1 kHz frequency, 0.330V offset, 5 $\mu$s width, 5 ns edge.
Scope: Ch1 vertical 0.200 V, trigger 0.3 V; Ch2 vertical 2V, time scale 4 $\mu$s.

MCA settings for all measurements:
  - 3 dB attenuation
  - 1024 ADC Gain
  - 10V maximum input voltage
  - The amplifier bias for the detector was set at 40V

Collected calibration spectra at 12 pulser voltages (15 s each), ranging from
0.035 V (0.32 MeV equivalent) to 0.575 V (5.32 MeV equivalent).
Determined centroid channel for each peak.

Calibration data:

% | Pulser Voltage (V) | Equiv. Energy (MeV) | Centroid Channel | Std (channels) |
% |---------------------|---------------------|------------------|----------------|
% | 0.035               | 0.324               | 59.04            | 0.85           |
% | 0.075               | 0.694               | 124.18           | 0.85           |
% | 0.125               | 1.157               | 208.89           | 1.59           |
% | 0.175               | 1.620               | 292.37           | 2.80           |
% | 0.225               | 2.082               | 375.56           | 1.62           |
% | 0.275               | 2.545               | 453.39           | 1.60           |
% | 0.325               | 3.008               | 542.05           | 4.30           |
% | 0.375               | 3.471               | 618.12           | 3.34           |
% | 0.425               | 3.933               | 708.89           | 1.60           |
% | 0.475               | 4.396               | 783.09           | 0.85           |
% | 0.525               | 4.859               | 875.60           | 6.20           |
% | 0.575               | 5.321               | 953.50           | 0.87           |

\begin{table}[htbp]
    \centering
    \caption{Energy calibration data}
    \label{tab:calibration-data}
    \begin{tabular}{cccc}
        \toprule
        Pulser Voltage (V) & Equiv. Energy (MeV) & Centroid Channel & Std (channels) \\
        \midrule
        0.035 & 0.324 & 59.04  & 0.85 \\
        0.075 & 0.694 & 124.18 & 0.85 \\
        0.125 & 1.157 & 208.89 & 1.59 \\
        0.175 & 1.620 & 292.37 & 2.80 \\
        0.225 & 2.082 & 375.56 & 1.62 \\
        0.275 & 2.545 & 453.39 & 1.60 \\
        0.325 & 3.008 & 542.05 & 4.30 \\
        0.375 & 3.471 & 618.12 & 3.34 \\
        0.425 & 3.933 & 708.89 & 1.60 \\
        0.475 & 4.396 & 783.09 & 0.85 \\
        0.525 & 4.859 & 875.60 & 6.20 \\
        0.575 & 5.321 & 953.50 & 0.87 \\
        \bottomrule
    \end{tabular}
\end{table}

A linear fit $E = a + b \cdot \text{ch}$ to the twelve calibration points yields
\begin{equation}
    a = -0.00568 \pm 0.02777\ \text{MeV}, \quad
    b = 0.00559 \pm 4.97 \times 10^{-5}\ \text{MeV/channel},
    \label{eq:calib-fit}
\end{equation}
with $R^2 = 0.9997$, confirming excellent linearity over the full 0.3--5.3\,MeV range.
The intercept is consistent with zero within its uncertainty, indicating no measurable
offset in the electronics chain. This fit is used in all subsequent energy calculations.

% Include calibration plot figure here
\begin{figure}[htbp]
    \centering
    \includegraphics[width=0.8\textwidth]{plots/Images/calibration-plot.png}
    \caption{Energy calibration: MCA channel number vs.\ equivalent energy (MeV).
    Linear fit with $R^2 = 0.9997$.}
    \label{fig:calibration}
\end{figure}

\subsection{Measurement Procedure}
\label{sec:measurement}

To keep the source-to-detector distance fixed across all measurements, we used the ``range-out'' position---the point at which alpha particles can no longer reach the detector at atmospheric pressure. We found this by slowly moving the source away from the detector until counts dropped to zero, which happened at a scale reading of $s = 10.8$\,cm. The source was locked there for all four gases. The physical source-to-detector distance follows from the chamber geometry as $D = (s - 6.58)\,\text{cm} = 4.22 \pm 0.13$\,cm.

With the source fixed, we swept the chamber pressure from $\sim$50\,mmHg up to near atmospheric and collected a 45-s MCA spectrum at each pressure point. We initially tried 15-s runs but the statistics were too poor to reliably locate the peak centroid, so all reported data use 45\,s. At each pressure we pulled the centroid channel from the spectrum and converted it to energy using the calibration from \S\ref{sec:calibration}. The effective path length at STP is then $X_\text{eff} = Dp/P_\text{atm}$, where $p$ is the chamber pressure and $P_\text{atm}$ is atmospheric pressure on the day of the measurement. Pressure readings carry an uncertainty of $\pm 0.1$\,mmHg.

\subsubsection{Air}
\label{sec:air_procedure}

We took 16 spectra in air at pressures from 54.8 to 765.1\,mmHg. At full atmospheric pressure no counts arrived at all, and at 721.2\,mmHg only 54 counts appeared over 45\,s---right at the edge of the range. Below 674.9\,mmHg the count rate settled into a stable plateau of roughly 1000--1200 counts per 45\,s. Because the count rate drops sharply near range-out, we added three extra points at 686.6, 700.8, and 711.9\,mmHg to better resolve that transition region.

\subsubsection{Helium}
\label{sec:he_procedure}

Helium has a much lower electron density than the diatomic gases, so alpha particles lose energy more slowly per centimeter and the range is much longer. We needed to go well above atmospheric pressure to see significant energy loss over our fixed distance, so we took 19 spectra from 50.9\,mmHg up to $\sim$2000\,mmHg. Even at the highest pressures the MCA peak only shifted down to around channel 624, meaning the particles were still arriving with substantial energy---we never observed range-out in helium. The spectrum at 50.9\,mmHg showed MCA clipping and was dropped from the analysis.

\subsubsection{Argon}
\label{sec:ar_procedure}

In argon we took 16 spectra from 50.4 to 782.0\,mmHg, with the source at the same $s = 10.8$\,cm position used for all other gases. Near the cutoff, at 770.8 and 759.6\,mmHg, the peak landed close to channel 47 where MCA clipping occurs, so those two points were excluded from the fits. At 782.0\,mmHg only a single count appeared over 45\,s, confirming the particles had essentially stopped.

\subsubsection{Nitrogen}
\label{sec:n2_procedure}

We collected 24 spectra in nitrogen from 50.8 to 782.3\,mmHg. Since air is roughly 78\% nitrogen, we expected similar behavior to air, and the range-out did fall at nearly the same pressure. To get good coverage of both the drop-off region and the plateau, we added extra points in the 250--440\,mmHg and 650--730\,mmHg ranges. Two data points---near 420 and 256\,Torr---were clear outliers relative to the neighboring measurements. We took several additional spectra at nearby pressures to confirm they were anomalous rather than a real feature, and excluded them from the fits.

%===============================================================================
\section{Data Analysis and Results}
\label{sec:results}
%===============================================================================

\subsection{Energy Calibration}
\label{sec:results-calibration}
The data collected during the energy calibration procedure (described in \S\ref{sec:calibration}) yields a linear relationship (shown in
Table~\ref{tab:calibration-data} and Fig.~\ref{fig:calibration}), which is what we expect given the specifications of the equipment provided by the manufacturer.
Performing a linear regression, we obtain the formula
\begin{align}
	E &= (5.59*10^{-3})C - 0.01 \\
	\sigma_E &= \sqrt{(0.03)^2 + ((5\times 10^{-5})C)^2 + ((5.59*10^{-3})\sigma_C)^2}
\end{align}

\subsection{Count Rate vs.\ Effective Distance}
\label{sec:countrate}

For each gas, the count rate (counts per second) was plotted as a function of
effective distance $X_\text{eff} = Dp/P_\text{atm}$, where $D$ is the
source-to-detector distance and $p$ is the chamber pressure.
The source-to-detector distance was $D = 4.22 \pm 0.13$\,cm for all gases.

For air, argon, and nitrogen, the count rate is approximately constant at low
$X_\text{eff}$ (low pressure), then drops sharply as $X_\text{eff}$ approaches
the particle range. This is characteristic behavior: alphas lose nearly all
their energy near the end of their path (Bragg peak), so a small increase
in effective path length near the range causes a large reduction in detected
count rate.

% [Tyler]: Describe the ALFA1 fit model used in Origin:
%   Equation: n0*(1 - integral(intgauss, 0, p, a, p0))
%   What is p0 physically? (mean range) What is 'a'? (related to straggling width?)
%   I think this section should explain the physical meaning of each fit parameter.

The count rate data were fit in OriginPro using a user-defined model (ALFA1)
based on the integral of a Gaussian, which accounts for range straggling.
% [Tyler]: Expand on the model derivation/justification.
The fit parameters for each gas are listed in Table~\ref{tab:countrate-fits}.

\begin{table}[htbp]
    \centering
    \caption{Count rate fit parameters (ALFA1 model) for air, argon, and nitrogen.
             $n_0$ is the plateau count rate, $p_0$ is the mean range, and $a$
             is a straggling parameter. % [Tyler]: confirm parameter definitions.
             Helium showed no drop in count rate over the measured range
             and is excluded from this table.}
    \label{tab:countrate-fits}
    \begin{tabular}{lccc}
        \toprule
        Gas & $n_0$ (s$^{-1}$) & $a$ & $p_0$ (cm) \\
        \midrule
        Air      & $23.83 \pm 0.33$ & $0.0919 \pm 0.0081$ & $3.925 \pm 0.004$ \\
        Argon    & $23.37 \pm 0.52$ & $0.0700 \pm 0.0095$ & $4.195 \pm 0.005$ \\
        Nitrogen & $23.28 \pm 0.27$ & $0.0870 \pm 0.0066$ & $3.943 \pm 0.003$ \\
        \bottomrule
    \end{tabular}
\end{table}

Helium was also measured, but no range-out was observed over the full range of
accessible pressures (up to approximately 2000\,mmHg, corresponding to
$X_\text{eff} \approx 11$\,cm). The count rate for helium showed no
statistically significant trend with $X_\text{eff}$ (linear fit slope
$0.200 \pm 0.099$\,s$^{-1}$\,cm$^{-1}$, consistent with zero). Helium is
therefore excluded from the direct range determination and is treated separately
via energy analysis only.

% TODO: Discuss peak broadening with increasing pressure -- why do the
% MCA peaks get wider as pressure increases? Answer?: more scattering collisions
% -> more energy straggling -> broader energy distribution at the detector.

\subsection{Energy Squared vs.\ Effective Distance}
\label{sec:e2}

If the energy loss per unit distance follows $dE/dx = k/E$ (as expected for heavy
charged particles in the relevant energy regime), then integrating from $E_0$ to zero
over the full range $R_0$ gives the range:
\begin{equation}
    R_0 = \frac{E_0^2}{2k},
    \label{eq:range}
\end{equation}
and the energy as a function of position:
\begin{equation}
    E^2 = E_0^2 - 2k \cdot X_\text{eff}.
    \label{eq:e2-linear}
\end{equation}
Thus $E^2$ should be a linear decreasing function of $X_\text{eff}$ with slope $-2k$
and $y$-intercept $E_0^2$. The range $R_0$ is then obtained as $R_0 = E_0^2 / (2k)$,
equivalently the $x$-intercept of the $E^2$ vs.\ $X_\text{eff}$ line.

$E^2$ vs.\ $X_\text{eff}$ was plotted for all four gases and fit with a linear
model (Figs.~\ref{fig:air-e2}--\ref{fig:he-e2}). % [Tyler]: verify figure labels match
The resulting fit parameters are given in Table~\ref{tab:e2-fits}.

\begin{table}[htbp]
    \centering
    \caption{Linear fit parameters for $E^2$ vs.\ $X_\text{eff}$.
             The slope equals $-2k$ and the $x$-intercept gives the range $R_0$.
             Helium values are extrapolated (range not directly observed).}
    \label{tab:e2-fits}
    \begin{tabular}{lcccc}
        \toprule
        Gas & $E_0^2$ (MeV$^2$) & Slope (MeV$^2$/cm) & $k$ (MeV$^2$/cm) & $R_0$ (cm) \\
        \midrule
        Air      & $29.95 \pm 0.45$ & $-8.473 \pm 0.227$ & $4.237 \pm 0.114$ & $3.535 \pm 0.109$ \\
        Argon    & $29.57 \pm 0.49$ & $-7.917 \pm 0.242$ & $3.959 \pm 0.121$ & $3.734 \pm 0.130$ \\
        Nitrogen & $29.68 \pm 0.44$ & $-8.368 \pm 0.228$ & $4.184 \pm 0.114$ & $3.547 \pm 0.110$ \\
        Helium   & $30.77 \pm 0.28$ & $-1.649 \pm 0.058$ & $0.825 \pm 0.029$ & $18.65 \pm 0.68$ \\
        \bottomrule
    \end{tabular}
\end{table}

% R_0 = intercept / |slope|, uncertainty via R_0 * sqrt((sigma_a/a)^2 + (sigma_b/b)^2)
% Air:  R_0 = 29.951/8.473 = 3.536 cm,  sigma = 3.536*sqrt((0.446/29.95)^2+(0.227/8.473)^2) = 0.108 cm
% Ar:   R_0 = 29.566/7.917 = 3.735 cm,  sigma = 3.735*sqrt((0.485/29.57)^2+(0.242/7.917)^2) = 0.130 cm
% N2:   R_0 = 29.678/8.368 = 3.547 cm,  sigma = 3.547*sqrt((0.443/29.68)^2+(0.228/8.368)^2) = 0.110 cm
% He:   R_0 = 30.766/1.649 = 18.66 cm,  sigma = 18.66*sqrt((0.282/30.77)^2+(0.058/1.649)^2) = 0.677 cm

The $E^2$ vs.\ $X_\text{eff}$ relationship is linear for all gases in the
well-measured energy region ($E \gtrsim 1$\,MeV). Data points at very low energies
(near the end of range) deviate from the linear fit; these correspond to
conditions where the alpha barely reaches the detector and the MCA peak channel
is less reliable.
% [Tyler]: Confirm whether those near-zero-energy points were included or
% excluded from the linear fit, and note any zero-count replacement used.

The intercept $E_0^2$ can be compared to the known Am-241 alpha energy:
$E_0 = 5.486$\,MeV gives $E_0^2 = 30.10$\,MeV$^2$. All four gases yield
intercepts consistent with this value within $1$--$2\sigma$, providing an
independent consistency check on the calibration.

% Sanity check: from air E vs Xeff quadratic fit, intercept = 5.485 +/- 0.031 MeV
% (extracted directly from Origin). Known Am-241 principal energy = 5.4857 MeV.
% Agreement is excellent.

\begin{figure}[htbp]
    \centering
    \includegraphics[width=0.85\textwidth]{plots/Images/air_count_rate.png}
    \caption{Count rate vs.\ $X_\text{eff}$ for alpha particles in air.
             The plateau at $\approx 24$\,s$^{-1}$ is followed by a sharp drop
             near the mean range $p_0 = 3.925$\,cm. The red curve is the ALFA1 fit.}
    \label{fig:air-countrate}
\end{figure}

\begin{figure}[htbp]
    \centering
    \includegraphics[width=0.85\textwidth]{plots/Images/air_e2.png}
    \caption{$E^2$ vs.\ $X_\text{eff}$ for alpha particles in air, with linear fit.
             Slope $= -8.473 \pm 0.227$\,MeV$^2$/cm, giving $k = 4.237$\,MeV$^2$/cm
             and extrapolated range $R_0 = 3.535 \pm 0.109$\,cm.}
    \label{fig:air-e2}
\end{figure}

\begin{figure}[htbp]
    \centering
    \includegraphics[width=0.85\textwidth]{plots/Images/ar_countrate.png}
    \caption{Count rate vs.\ $X_\text{eff}$ for alpha particles in argon.
             Mean range $p_0 = 4.195$\,cm from ALFA1 fit.}
    \label{fig:ar-countrate}
\end{figure}

\begin{figure}[htbp]
    \centering
    \includegraphics[width=0.85\textwidth]{plots/Images/ar_e2.png}
    \caption{$E^2$ vs.\ $X_\text{eff}$ for argon.
             Slope $= -7.917 \pm 0.242$\,MeV$^2$/cm, $R_0 = 3.734 \pm 0.130$\,cm.}
    \label{fig:ar-e2}
\end{figure}

\begin{figure}[htbp]
    \centering
    \includegraphics[width=0.85\textwidth]{plots/Images/n2_count.png}
    \caption{Count rate vs.\ $X_\text{eff}$ for alpha particles in nitrogen.
             Mean range $p_0 = 3.943$\,cm. The outlier near $X_\text{eff} \approx 2$\,cm
             % [Tyler]: explain this outlier -- was it excluded from the fit?
             lies well below the plateau.}
    \label{fig:n2-countrate}
\end{figure}

\begin{figure}[htbp]
    \centering
    \includegraphics[width=0.85\textwidth]{plots/Images/n2_e2.png}
    \caption{$E^2$ vs.\ $X_\text{eff}$ for nitrogen.
             Slope $= -8.368 \pm 0.228$\,MeV$^2$/cm, $R_0 = 3.547 \pm 0.110$\,cm.}
    \label{fig:n2-e2}
\end{figure}

\begin{figure}[htbp]
    \centering
    \includegraphics[width=0.85\textwidth]{plots/Images/He_count.png}
    \caption{Count rate vs.\ $X_\text{eff}$ for helium. No drop-off is observed;
             the linear fit slope ($0.20 \pm 0.10$\,s$^{-1}$\,cm$^{-1}$) is
             consistent with zero. The alpha range in helium exceeds the maximum
             achievable $X_\text{eff}$ at the pressures measured.}
    \label{fig:he-countrate}
\end{figure}

\begin{figure}[htbp]
    \centering
    \includegraphics[width=0.85\textwidth]{plots/Images/he_e2.png}
    \caption{$E^2$ vs.\ $X_\text{eff}$ for helium. Linear fit over the measured
             range gives slope $= -1.649 \pm 0.058$\,MeV$^2$/cm,
             extrapolating to $R_0 = 18.65 \pm 0.68$\,cm.}
    \label{fig:he-e2}
\end{figure}

%===============================================================================
\section{Discussion and Conclusions}
\label{sec:conclusion}
%===============================================================================

% TODO: Discuss results in context.
%
% - How well does E^2 vs X_eff linearity hold? Where does it break down?
% - Meaning of k varying with energy (Bethe formula predicts log dependence, not pure 1/E)
% - Comparison of range in different gases to theoretical expectations
% - Discuss straggling, Bragg peak
% - Sources of systematic error (clipping at low channels, vacuum control issues with He)
% - Comparison to example report's ~10% discrepancy with literature

\nocite{*}
\bibliographystyle{alpha}
\bibliography{refs}
\end{document}
